\let\negmedspace\undefined
\let\negthickspace\undefined
\documentclass[journal,12pt,twocolumn]{IEEEtran}
\usepackage{cite}
\usepackage{amsmath,amssymb,amsfonts,amsthm}
\usepackage{algorithmic}
\usepackage{graphicx}
\usepackage{textcomp}
\usepackage{xcolor}
\usepackage{txfonts}
\usepackage{listings}
\usepackage{enumitem}
\usepackage{mathtools}
\usepackage{gensymb}
\usepackage[breaklinks=true]{hyperref}
\usepackage{tkz-euclide} % loads  TikZ and tkz-base
\usepackage{listings}
\usepackage{gvv}


        

\newtheorem{theorem}{Theorem}[section]
\newtheorem{problem}{Problem}
\newtheorem{proposition}{Proposition}[section]
\newtheorem{lemma}{Lemma}[section]
\newtheorem{corollary}[theorem]{Corollary}
\newtheorem{example}{Example}[section]
\newtheorem{definition}[problem]{Definition}
\newcommand{\BEQA}{\begin{eqnarray}}
\newcommand{\EEQA}{\end{eqnarray}}
\newcommand{\define}{\stackrel{\triangle}{=}}
\theoremstyle{remark}
\newtheorem{rem}{Remark}

%\bibliographystyle{ieeetr}
\begin{document}
%

\bibliographystyle{IEEEtran}


\vspace{3cm}

\title{
%	\logo{
Assignment-2

\large{EE1205 : Signals and Systems}

Indian Institute of Technology Hyderabad
%	}
}
\author{Chirag Garg

(EE23BTECH11206)
}	





\maketitle

\newpage



\bigskip

\renewcommand{\thefigure}{\theenumi}
\renewcommand{\thetable}{\theenumi}


\section{Question 11.9.1 (5)}
\vspace{0.5cm}
\begin{flushleft}
 Write the first five terms of the sequence whose $n^{th}$ \text{term is} : $a_n = (-1)^{n-1}5^{n+1}$.
\end{flushleft} 


\vspace{0.8cm}


\section{Solution} 
\begin{flushleft}

\end{flushleft}
The $n^{th}$ term of sequence is: $a_n = (-1)^{n-1}5^{n+1}$\\
On substituting n = 0, 1, 2, 3 and 4, we get the first five terms as,
\begin{flushleft}

\text{1st term} : $a_0 = (-1)^{0-1}5^{0+1} =  -5 $\\
\text{2nd term} : $a_1 = (-1)^{1-1}5^{1+1} = 25$ \\
\text{3rd term} : $a_2 = (-1)^{2-1}5^{2+1} =  -125$ \\
\text{4th term} : $a_3 = (-1)^{3-1}5^{3+1} = 625$\\
\text{5th term} : $a_4 = (-1)^{4-1}5^{4+1} =  -3,125$\\ 
\end{flushleft}
Hence, the required terms are -5, 25, –125, 625, –3125 .


\end{document}