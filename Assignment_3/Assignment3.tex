\let\negmedspace\undefined
\let\negthickspace\undefined
\documentclass[journal,12pt,twocolumn]{IEEEtran}
\usepackage{cite}
\usepackage{amsmath,amssymb,amsfonts,amsthm}
\usepackage{algorithmic}
\usepackage{graphicx}
\usepackage{textcomp}
\usepackage{xcolor}
\usepackage{txfonts}
\usepackage{listings}
\usepackage{enumitem}
\usepackage{mathtools}
\usepackage{gensymb}
\usepackage[breaklinks=true]{hyperref}
\usepackage{tkz-euclide} % loads  TikZ and tkz-base
\usepackage{listings}
\usepackage{gvv}


        

\newtheorem{theorem}{Theorem}[section]
\newtheorem{problem}{Problem}
\newtheorem{proposition}{Proposition}[section]
\newtheorem{lemma}{Lemma}[section]
\newtheorem{corollary}[theorem]{Corollary}
\newtheorem{example}{Example}[section]
\newtheorem{definition}[problem]{Definition}
\newcommand{\BEQA}{\begin{eqnarray}}
\newcommand{\EEQA}{\end{eqnarray}}
\newcommand{\define}{\stackrel{\triangle}{=}}
\theoremstyle{remark}
\newtheorem{rem}{Remark}

%\bibliographystyle{ieeetr}
\begin{document}
%

\bibliographystyle{IEEEtran}


\vspace{3cm}

\title{
%	\logo{
Discrete

\large{EE1205 : Signals and Systems}

Indian Institute of Technology Hyderabad
%	}
}
\author{Chirag Garg

(EE23BTECH11206)
}	





\maketitle

\newpage



\bigskip

\renewcommand{\thefigure}{\theenumi}
\renewcommand{\thetable}{\theenumi}


\section{Question 11.9.5 (18)}
\vspace{0.5cm}
\begin{flushleft}
If $a$ and $b$ are the roots of $x^{2} -3x + p = 0$ and $c$ , $d$ are roots of $x^{2} - 12x + q = 0$ where $a,b,c,d$ form a G.P. Prove that $(q+p) : (q-p)$ = 17:15 .
\end{flushleft}  


\vspace{0.8cm}


\section{Solution} 
\begin{flushleft}
\end{flushleft}
Given $a$ and $b$ are the roots of $x^{2} – 3x + p = 0$
So, we have :
\begin{align}
a + b = 3\\
 ab = p 
\end{align}

Also, $c$ and $d$ are the roots of $x^{2} – 12x + q = 0$ , so,
\begin{align}
c + d = 12 \\
cd = q
 \end{align}  
And given $a, b, c, d$ are in G.P.
Let’s take $a = x, b = xr, c = xr^{2}, d = xr^{3}$ ,where $x$ and $r$ are first term and common ratio of the G.P. respectively.\\
From (1) and (3) , we get , 
\begin{align}
x + xr = 3 \\
x(1+r)=3
\end{align}
 
 And ,
 \begin{align}
xr^{2} + xr^{3} = 12 \\
xr^{2}(1+r) = 12
\end{align}
On dividing eq. (5) and eq. (6), we get
\begin{align}
\dfrac{xr^{2}(1+r)}{x(1+r)} = \dfrac{12}{3} \\
r^{2} = 4 \\
r = \pm2
\end{align}

When r = 2, x = 3/(1 + 2) = 3/3 = 1

When r = -2, x = 3/(1 – 2) = 3/-1 = -3

\vspace{0.5cm}
Case 1 :
When $r = 2$ and $x = 1$ \\
\begin{align}
p &= ab \\
p&=2 \\
q&= cd \\
q&=32 \\
\dfrac{q+p}{q-p} &= \dfrac{32 + 2}{32 - 2} \\
&=\dfrac{17}{15}
\end{align}
Case 2 :
When $r = -2$ and $x = -3$ \\
\begin{align}
p &= ab \\
p&=-18 \\
q&= cd \\
q&=288 \\
\dfrac{q+p}{q-p} &= \dfrac{288-18}{288+18} \\
&=\dfrac{135}{153}
\end{align}
Hence , case 1 satisfies the condition .
\end{document}